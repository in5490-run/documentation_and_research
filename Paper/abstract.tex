\begin{abstract}
	In this paper we present an experimental framework for geolocalizing Unmanned Aerial Vehicles(UAV) using orthogonal imagery of the landscape below the drone. 
	The framework exploits the use of machine learning, Monte Carlo localization and computer vision, along with the enormous amount of satellite imagery and orthophotos available, to approximate the position of the UAV. 
	It exploits natural and man-made topographical features as fingerprints obtained from the scene below, to gain a more precise position without the use of a Global Positioning System(GPS). 
	This is advantageous, and in some applications critical, in GPS-denied areas as well as providing additional precision to existing GPS-dependent systems.
	End goal of the framework is to represent a solution adequately invariant to seasons, weather, lighting and other environmental changes. 
	Geolocalization is performed by an somewhat optimized version of Monte Carlo localization(MCL) on synthetic images made invariant to disturbances such as lighting and seasons. The MCL is primarily optimized by applying particles only on the stable features extracted. And some other cool shit.
\end{abstract}
